El encapsulamiento es ver a los objetos como capsulas o cajas negras que interectuan entre si, sin conocerse demasiado, y asi y todo poder resolver problemas utilizando esta interaccion entre
objetos

Existen dos grandes definiciones para encapsulamiento
1° Deberiamos tomar ciertos atributos y ciertos metodos que hagan a un tipo particular de objeto y envolverlos o encapsularlos en una unica unidad llamada CLASE de esta manera favoreceriamos  la reutilizacion
de otros proyectos de ese tipo de datos(es decir,que dejamos todo guardado en una clase y luego esa la clase la podremos usar para otros proyectos,por ende podremos usar sus atributos y sus metodos como por ejemplo la clase String)
2°El encapsulamiento nos habla de que los objetos deberian mantener su integridad OCULTAR sus detalles de implementacion OCULTAR su estado interno para que nadie pueda cambiarlo asi sin mas a su voluntad


            Persona
    ------------------------
    -DNI:String
    -nombre:String
    -apellido:String
    -calle:String                       DIAGRAMA UML
    -altura:int
    -barrio:String
    ------------------------
    +domicilioComoCadena():String
    +nombreCompleto():String

    Los atributos son DNI,nombre,apellido,calle,altura,barrio
    Los metodos son domicilioComoCadena,nombreCompleto

    2° clase

            Empresa
    --------------------------
    -razonSocial:String
    -CUIT:String
    -calle:String
    -altura:int
    -barrio:String
    --------------------------
    +domicilioComoCadena():String
    +contratar():void
    pagarSalarios():void

    Notamos que hay ciertos atributos y metodos repetidos en ambas clases,si yo noto que dicho metodos y atributos se repiten ademas que no parecen ser algo exclusivo de esas clases,si no que podemos verlo como una unidad independiente
    entonces la idea seria que aparezca una nueva clase que encapsule dichos atributos y metodos,asi poder hacer reutilizacion

        Domicilio
---------------------------------
    -calle:String
    -altura:int
    -barrio:String
---------------------------------
domicilioComoCadena():String


integridad

Otra cuestion importante es que los objetos garanticen su integridad,es decir que no queden en un estado inconsistente

        CuentaBancaria
    ------------------------
    CBU:String
    alias:String
    saldo:double
    ------------------------                            miCuenta.saldo= -123
    obtenerSaldo():double                               miCuenta.CBU="Hola que tal"
    depositar(double):void
    extraer(double):void
    generarCbu():String
    saldoSuficiente(double):boolean    
    
Podemos ver como ejemplo a la clase cuenta bancaria,donde sus atributos y metodos estan expuestos,eso quiere decir que,si vamos a cualquier otra clase creamos un objeto cuenta Bancaria y lo guardamos
en la variable mi cuenta,yo podre guardar en mi cuenta un saldo negativo o cualquier CBU,lo cual no nos daria ningun tipo de error,pero para la funcionalidad que yo quiero del programa un saldo negativo no es correcto
ademas en la vida real,nosotros no podemos establecer asi sin mas el saldo,este debe pasar por distintas operaciones,por lo tanto debemos garantizar que esto no pueda hacerse y para ello se utiliza 
el concepto de Visibilidad

Encapsular el estado interno 

A partir de ahora casi de forma obligatoria para garantizar el encapsulamiento es colocar a todos los atributos de una clase la visibilidad privada,al ser privada los atributos,el unico capaz de poder
consultarlo directamente es la propia clase que los declaro y asi entonces cualquier otra clase no podra saber ni siquiera que existen


CuentaBancaria
------------------------
CBU:String
alias:String
saldo:double
------------------------                            x=miCuenta.saldo
obtenerSaldo():double                               miCuenta.saldo=miCuenta.saldo+100
depositar(double):void
extraer(double):void
generarCbu():String
saldoSuficiente(double):boolean


CuentaBancaria
------------------------
-CBU:String
-alias:String
-saldo:double
------------------------                            x=miCuenta.obtenerSaldo()
+obtenerSaldo():double                              miCuenta.depositar(100) en el metodo depositar,obligamos que el saldo sea positivo
+depositar(double):void
+extraer(double):void
-generarCbu():String
-saldoSuficiente(double):boolean

TODOS LOS ATRIBUTOS DEBERAN TENER VISIBILIDAD PRIVADA


Ocultar detalles de implementacion

No todos los metodos deben ser publicos si no ,solamente aquellos que yo quiero exponer para que los objetos de otro tipos interactuen entre si para resolver problemas (+ es publico)