//Que es un objeto -- Es una identidad que disponga de un Estado,Comportamiento y una identidad

//Cada objeto tiene un conjunto de propiedades los cuales se denominan atributos,por ejemplo los autos los cuales tienen infinidad de atributos reconocibles
//sin embargo se realiza un proceso denominado ABSTRACCION

//ABSTRACCION nos permite saber analizar primero,que atributo nos interesa modelar en el codigo,para que de solucion al problema,de esta manera no es necesario detallar todos los atibutos

//Los ATRIBUTOS de un objeto definiran basicamente las propiedades que caracterizan al objeto

//El ESTADO de un objeto,viene dado por el valor de esos atributos,para un cierto instante,ya que estos pueden variar para un mismo objeto,en el transcurso del tiempo

//COMPORTAMIENTO de un objeto,son las  operaciones(acciones) que puede realizar un objeto a traves de sus metodos

//La IDENTIDAD de un objeto,es lo que nos permite diferenciar un objeto con otros
//Por ejemplo ,nada impide que yo pueda tener varios autos y que dos de ellos tengan el mismo estado,es decir, mismo color,marca,kilometraje,etc,hasta tambien las mismas operaciones
//por que son capaces de acelerar,frenar,encender,etc.Entonces,¿Como hago para pedirle operaciones a un auto en especifico sin afectar a los demas?
//para ello,cada vez que se cree un objeto JAVA le asigna una dirreccion,es decir una referencia en la memoria,cada objeto tendra una dirreccion en la memoria para siempre,nosotros 
//si queremos podemos ver la dirreccion de la memoria,pero no asi manipularlo,por ende podemos hablar a un objeto cambiar sus atributos o metodos sin afectar a los demas objetos 

