En javaScript es mas sencillo tener objetos,ya que solamente debemos abrir un par de llaves y dentro de ellas escribir los atributos junto a sus valores,y luego los mertodos,
pero este tipo de enfoque nos llevara a un escenario en donde tenermos por ejemplo 4 tipos de autos(objetos) y algunos responderan al color otros no,alguno van a saber que tienen
kilometros recorridos otros no,etc.

Esto derivara a posibles errores muy dificiles de detectar,para evitar entonces estos tipo de inconcistencias entre objetos del mismo tipo es que aparece en JAVA el concepto de CLASE


Que es una clase?-Para explicar eso usaremos la analogia con un plano de una casa,previamente a que la casa exista fisicamente nosotros solemos elaborar un boceto o plano mostrando como
sera esa casa,de la misma forma entonces nosotros antes de decir que tendremos objetos de tipo autos,vamos a modelar lo que es la Clase autos

Modelar una clase siginifica explicarle al lenguaje  que atributos y que metodos va tener cada objeto de tipo auto que en el futuro vaya a crearse

Esto significa entonces que crear una clase no significa estar creando un objeto, si no estar creando un MOLDE o PLANTILLA que nos va a permitir tener a futuro objetos,pero todos consistentes 
es decir todos con el mismo lote de atributos aunque esos valores puedan variar y todos con el mismo conjunto de comportamientos a traves de sus metodos

Por ende los objetos,se los puede resumir que es la INSTANCIA DE UNA CLASE 